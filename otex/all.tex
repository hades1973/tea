.s.1. 某次考试共有68题,考试时间为70分钟,如果每题用的时间基本相同并且刚好能在70分钟内做完,
那么做到第35题时的合理时间应该在第几分钟?(\quad).
A. 33-34分钟
B. 34-35分钟
C. 35-36分钟
D. 36-37分钟

.s.2. 有甲乙两个仓库,从甲仓库拿出6吨货物给乙仓库,两个仓库的重量就相等。从乙仓库拿出6吨货物给甲仓库,甲仓库的重量就是乙仓库的2倍。请问两个仓库一共重多少吨?(\quad)
A. 12
B. 24
C. 48
D. 72

.s.3. 两杯重量一样的糖水,一杯糖水的比是1: 3, 另一杯糖水的比是1:4,现在将两杯糖水混合在一起,则糖水之比是(\quad)
A. 2:7
B. 1:7
C. 3:9
D. 9:31

.s.4. 爸爸在2017年4月以每股4元的价钱钱买进了1000股首创股份的股票,在6月份以每股7元钱全部卖出。到了8月份又以每股6元钱买入该股1000股,
到了10月份以每股5元钱卖出该股。问爸爸在首创股份这只股票上总共赚了多少钱?(\quad)
A. 2000元
B. 1500元
C. 4000元
D. 4500元

.s.5 6的因数有1、2、3、6。它们的关系是1+2+3=6。像6这样的数叫做完全数。下面属于完全数的是(\quad)。
A. 8
B. 24
C. 28
D. 48

.s.6 a是比0大的最小自然数,b是质数中的偶数,c是最小的奇质数,d和c加在一起等于39,那么$a+b\times c + d\times(b+c) = $~(\quad)。
A. 97
B. 187
C. 45
D. 66

.s.7 甲乙丙三个数,甲比乙是2:5,乙比丙是4:7,甲乙丙三数和是126,问甲是多少?(\quad)
A. 24
B. 37
C. 16
D. 18

.s.8. 12月有5个星期二,那么下一年1月一定会有5个(\quad)
A. 星期一
B. 星期二
C. 星期三
D. 星期五

.s.9. 时代中学要从甲乙丙丁4人中选出1名三优学生参加夏令营。
甲说:“我们四个人中每人至少得一个优,并且只有一个三优学生”。
乙说:“我和甲的数学成绩一样,我和丙的语文成绩一样”。
丙说:“我和丁中只有一个人的语文成绩是优”。
丁说:“我们四人中有三个人语文成绩是优,两个人数学得优,一个人英语是优”。
如果他们都没有说谎,那么这个三优学生是(\quad)。
A. 甲
B. 乙
C. 丙
D. 丁

.s.10. 将一根木头锯成六段需要3分钟,则锯成九段需要几分钟?(\quad)。
A. 4分钟
B. 4.8分钟
C. 5分钟
D. 5.2分钟

.s.11. 如图大小两个正方形并在一起,已知小正方形边长为6cm,则阴影部分面积等于(\quad)$cm^2$
.fig. fig-shidai-1.png
A. 12
B. 16
C. 18
D. 36

.s.12. 有一个等腰梯形,它的其中三条边的长分别为65cm,35cm,15cm,并且它的下底最长,这个等腰梯形的周长是(\quad)
A. 115
B. 130
C. 150
D. 180

.s.13. 若$x=3,b=\frac{1}{2}$~,那么$2ab^2=$(\quad)
A. 3/2
B. 1/2
C. 1
D. 0

.s.14. 根据下图的规律,问号应该等于多少(\quad)
A. 738
B. 720
C. 550
D. 500

.s.15. 小明和小红同事在校门口出发回家,经过7分钟两人同时到家,小明平均速度为每分钟45米,小红平均速度为每分钟35米,
小明与小红家的距离不可能是(\quad)?
A. 60
B. 90
C. 300
D. 500

.s.16. 两个不同的自然数的最小公倍数是12,问有几种可能性?
A. 3
B. 6
C. 7
D. 8

.s.17. 桌面上放着印有A、B、C、D的四张卡片,老师让小明和小红随意取两张,则A、B两张卡恰好被同一个人拿走的可能性是(\quad)
A. $\frac{1}{10}$
B. $\frac{1}{5}$
C. $\frac{1}{3}$
D. $\frac{1}{2}$

.s.18. 下面哪个包背起来最舒服(\quad)
A. 背带宽为2厘米的单肩包
B. 背带宽为3厘米的单肩包
C. 背带宽为1厘米的双肩包
D. 背带宽为2厘米的双肩包

.s.19. 七分之三化为小数后的小数点后2018位是几?(\quad)
A. 2
B. 7
C. 4
D. 8

.s.20. 某抽奖活动规则如下:箱子里有三种不同图案的卡片各多个,每人可任意抽两张卡片,两张卡片如果是同一种图案则中奖。
如果让你派人去抽奖,至少要派几个人才能保证一定会中奖?(\quad)
A. 6
B. 7
C. 8
D. 9

.s.21. 甲仓库的粮食比乙仓库多10\%,乙仓库的粮食比丙仓库少10\%,问三个仓库哪一个仓库的粮食最多?(\quad)
A. 甲
B. 乙
C. 丙
D. 丁

.s.22. 记$S_1, S_2, S_3$为下图中1、2、3三个三角形的面积,请问下面关系正确的是(\quad)
A. $S_1 = S_2 = S_3$
B. $S_1 > S_2 = S_3$
C. $S_1 > S_2 > S_3$
D. $S_1 < S_2 = S_3$


.c.1. 为了去除果园里的害虫,需要使用DDT和水混合起来配制杀虫药水。DDT与水的比例为1:49时杀虫效果最好。现打算配制药水25千克,需DDT多少?
\vspace{4cm}

.c.2. 银行是进行存放贷业务的机构。储户将钱存入银行,这个钱称之为本金。经过一年后储户取钱时,银行不仅要支付给储户本金,还要支付一定的利息。
单位本金一年获取的利息称之为年利率,即
\[
	\text{年利率} = \frac{\text{利息}}{\text{本金}}
\]
如果婷婷将600元钱存入银行,整存整取两年,年利率按2\%计算,到期后婷婷能取到多少钱?
\vspace{4cm}

.c.3. 某商品降价20\%后,由于销量大增,现在想恢复原价,则价格应该提高~\blank~\%.
\vspace{4cm}

.c.4. 琳琳看《西游记》,上午看了50页,相当于下午看的页数的$\cfrac{7}{8}$又1页,琳琳今天共看了~\blank~页书。
\vspace{4cm}

.c.5. 我们看到的所有商店里的商品都是从工厂生产出来的。商店从工厂买来商品的价钱称之为进价或成本价。
商家为了赚钱,一般情况下,他们将商品以高于进价的价格卖给我们,从而能赚取一定的利润。
但是如果商品不好卖(通常称之为滞销),商家就会以低于进价的价格卖给我们,此时他会赔钱,称之为亏损。
单位进价赚取的利润称之为利润率,即
\[
	\text{利润率} = \frac{\text{利润}}{\text{进价}}
\]
现在平安药店以每斤60元的价格购进一批药材,按50\%的利润率定价销售,当卖出这批药材的80\%时,药房获利1440元。
问平安药房一共购进了多少斤药材?
\vspace{4cm}
