\documentclass[12pt]{article}
\usepackage{ctex}
\usepackage{lastpage}
\usepackage{layout}
\usepackage{color}
\usepackage{placeins}
\usepackage{ulem}
\usepackage{titlesec}
\usepackage{multirow}
\usepackage{graphicx}
\usepackage{float}
\usepackage{colortbl}
\usepackage{listings}
\usepackage{indentfirst}
\usepackage{fancyhdr}
\usepackage{amsmath,amssymb}
\usepackage{tikz}
\usepackage[a4paper,hmargin={2.18cm,2.18cm},vmargin={2.54cm, 2.54cm}]{geometry}
\usepackage{wallpaper}

%------------------------------设置字体大小------------------------%  
\newcommand{\chuhao}{\fontsize{42pt}{\baselineskip}\selectfont}     %初号  
\newcommand{\xiaochuhao}{\fontsize{36pt}{\baselineskip}\selectfont} %小初号  
\newcommand{\yihao}{\fontsize{28pt}{\baselineskip}\selectfont}      %一号  
\newcommand{\erhao}{\fontsize{21pt}{\baselineskip}\selectfont}      %二号  
\newcommand{\xiaoerhao}{\fontsize{18pt}{\baselineskip}\selectfont}  %小二号  
\newcommand{\sanhao}{\fontsize{15.75pt}{\baselineskip}\selectfont}  %三号  
\newcommand{\sihao}{\fontsize{14pt}{\baselineskip}\selectfont}%     四号  
\newcommand{\xiaosihao}{\fontsize{12pt}{\baselineskip}\selectfont}  %小四号  
\newcommand{\wuhao}{\fontsize{10.5pt}{\baselineskip}\selectfont}    %五号  
\newcommand{\xiaowuhao}{\fontsize{9pt}{\baselineskip}\selectfont}   %小五号  
\newcommand{\liuhao}{\fontsize{7.875pt}{\baselineskip}\selectfont}  %六号  
\newcommand{\qihao}{\fontsize{5.25pt}{\baselineskip}\selectfont}    %七号  
%-------------------------------------------------------------------------
\newcommand{\blank}[1]{\makebox[8cm][l]{\uline{\hfill #1 \hfill}}}
\newcommand{\blankk}[1]{\makebox[3cm][l]{\uline{\hfill #1 \hfill}}}

%------------------------------表格列简写---------------------------%  
\newcommand{\firstC}[1]{\begin{minipage}[t]{.25\textwidth}{\vspace{0pt}#1}\end{minipage}}
\newcommand{\secondC}[1]{\begin{minipage}[t]{.65\textwidth}{\vspace{0pt}#1\vspace{.5cm}}\end{minipage}}
\newcommand{\thirdC}[1]{\parbox[c]{10cm}{#1}}
%-------------------------------------------------------------------%

\pagestyle{empty}

\begin{document}

%%%%%%%%%%%%%%%%%%%%%%%%%%%%%%%%%%%%%%%%%%%%%%%%%%%%%%%%%%%%%%%%%%%%%%%%%%%%
%% 任务书封面
%%%%%%%%%%%%%%%%%%%%%%%%%%%%%%%%%%%%%%%%%%%%%%%%%%%%%%%%%%%%%%%%%%%%%%%%%%%%
\resizebox{3.74cm}{2.78cm}{\includegraphics{logo1.png}}
\resizebox{10.9cm}{2.86cm}{\includegraphics{logo2.png}}
\par{\vspace{4cm}}

\begin{center}
{\heiti \xiaochuhao 混凝土结构设计}\\\vspace{.7cm}
{\heiti \yihao 课程报告任务书}
\end{center}
\newpage


%%%%%%%%%%%%%%%%%%%%%%%%%%%%%%%%%%%%%%%%%%%%%%%%%%%%%%%%%%%%%%%%%%%%%%%%%%%%
%% 任务书内容
%%%%%%%%%%%%%%%%%%%%%%%%%%%%%%%%%%%%%%%%%%%%%%%%%%%%%%%%%%%%%%%%%%%%%%%%%%%%
\noindent {\heiti \zihao{4} 一、报告选题}
\begin{enumerate}
\item 装配式混凝土结构
\item 现代木结构
\item 中国古代木结构
\end{enumerate}
\indent 本任务书只给出报告的大方向,具体内容需各组自行讨论决定。
可以涉及建筑设计、结构设计、施工技术、施工组织、各种结构形式,甚至可以是某个案列的研究。
报告涉及的内容既不要过大也不要过小。最终提交纸质报告书不少于60页,课堂演讲至少20分钟不超过30分钟。
\par{}
\vspace{.5cm}

\noindent {\heiti \zihao{4} 二、要求}
\begin{enumerate}
\item 组长做好协调、分工。
\item 收集关于课题的图书、论文、图片、视频等资料。
\item 小组各成员针对报告大题目及自己阅读的内容相互之间进行交流。组长负责组织讨论会(至少三次)、
      每次均应拍摄照片四至五张,同时做好小组讨论的音频录制,然后整理出讨论记录。
\item 根据讨论结果拟定报告具体题目、内容,大纲。
\item 分工合作撰写报告及演讲幻灯片
\item 给全班同学做报告。
\item 整理成果提交给任课教师。
\end{enumerate}

\noindent {\heiti \zihao{4}  三、提交成果}
\begin{enumerate}
\item 打印书面报告一份。
\item 打印组长对各组员的表现报告一份(里面应真实、公平评价各组员的贡献)。
\item 光盘一张(此前收集的资料、拍摄照片、录制的音频以及正式版的书面报告、演讲幻灯片等文件)。
\end{enumerate}

\newpage
\noindent {\heiti \zihao{4}  四、分组}
\par{}\vspace{.5cm}
按班级分组,由于1班人数较多,所以分为两组:\vspace{.5cm}

\begin{tabular}{lc} 
	题\hspace{1.1cm}目:           & 学生名单 \\ \hline
	\firstC{装配式混凝土结构、木结构} & \secondC{  林楠\  黄俊亮\  庄志威\  李广智\  陈凡楠\  文豪\  游朝裕\  陈泽川\  章鹏山\  许泳森\  林张杰\  邓万枝\  闫朕\  郑攀攀\ } \\
	\firstC{装配式混凝土结构} & \secondC{  赵楚楚\  林永銮\  陈凌凡\  张晓捷\  沈泽林\  郑君松\  吕培榕\  占德炜\  林杰\ } \\
	\firstC{装配式混凝土结构} & \secondC{  刘蔚文\  翁丽丽\  林新剑\  谭毅\  骆韬\  陈淅\  许智军\  李嘉荣\  董建辉\  李昊洋\ } \\
	\firstC{现代木结构} & \secondC{  刘锦\  郭芳忠\  魏宁\  陈有官\  邱伟鹏\  张万松\  王赟\ } \\
	\firstC{中国古代木结构} & \secondC{  冯凌俊\  庄凌展\  王松明\  周培松\  朱年志\  牛昊\  张锦城\  张祥亿\  黄翔\ } \\
\end{tabular}\\

\end{document}
